\documentclass[12pt,a4paper]{report}
\usepackage[T2A]{fontenc}
\usepackage[utf8]{inputenc}
\usepackage[russian]{babel}
\usepackage{graphicx, setspace, hyperref}

\usepackage[
top = 1.25cm, 
bottom = 2.0cm]{geometry}

\begin{document}
\begin{titlepage} 
	\centering
    % HEADER
	{
        \scshape
        Федеральное государственное автономное образовательное учреждение высшего образования
        \par
        \textbf{«Научно-образовательная корпорация ИТМО»}
        \par
        \vspace*{1cm}
        Факультет Программной Инженерии и Компьютерной Техники
        \par
    }
    % LOGO
    \vspace*{0.6cm}
    \includegraphics[width=\textwidth]{logo.png}
    % LAB INFO
    {
        \Large
        \textbf{Лабораторная работа по веб-программированию №2}
        \par
        \normalsize
        \vspace*{0.75cm}
        \textbf{Вариант 90820}
        \par
    }
    \vfill
    % СREDITS
    \hfill\begin{minipage}{\dimexpr\textwidth-7.8cm}
        \textbf{Выполнил:}\par
        Степанов Арсений Алексеевич\par
        \vspace*{0.15cm}
        \textbf{Группа:}\par
        P3209\par
        \vspace*{0.15cm}
        \textbf{Преподаватель:}\par
        Кулинич Ярослав Вадимович\par
    \end{minipage}
    \vfill
    Санкт-Петербург, \the\year{}г.
\end{titlepage}  
\section*{Цели}
Познакомиться с Java EE, изучить базовые понятия и концепции, научиться работать с сервлетами и возможностями, предоставляемыми данной технологией
\section*{Задание}
Разработать веб-приложение на базе сервлетов и JSP, определяющее попадание точки на координатной плоскости в заданную область \\
\hfill\break
Приложение должно быть реализовано в соответствии с шаблоном MVC и состоять из следующих элементов:
\begin{itemize}
    \item \textbf{ControllerServlet}, определяющий тип запроса, и, в зависимости от того, содержит ли запрос информацию о координатах точки и радиусе, делегирующий его обработку одному из перечисленных ниже компонентов. Все запросы внутри приложения должны передаваться этому сервлету (по методу GET или POST в зависимости от варианта задания), остальные сервлеты с веб-страниц напрямую вызываться не должны
    \item \textbf{AreaCheckServlet}, осуществляющий проверку попадания точки в область на координатной плоскости и формирующий HTML-страницу с результатами проверки. Должен обрабатывать все запросы, содержащие сведения о координатах точки и радиусе области.
    \item \textbf{Страница JSP}, формирующая HTML-страницу с веб-формой. Должна обрабатывать все запросы, не содержащие сведений о координатах точки и радиусе области.
\end{itemize}
Разработанная страница JSP должна содержать:
\begin{enumerate}
    \item "Шапку", содержащую ФИО студента, номер группы и номер варианта
    \item Форму, отправляющую данные на сервер
    \item Набор полей для задания координат точки и радиуса области в соответствии с вариантом задания
    \item Сценарий на языке JavaScript, осуществляющий валидацию значений, вводимых пользователем в поля формы
    \item Интерактивный элемент, содержащий изображение области на координатной плоскости
    \item Таблицу с результатами предыдущих проверок. Список результатов должен браться из контекста приложения, HTTP-сессии или Bean-компонента в зависимости от варианта
\end{enumerate}
Страница, возвращаемая AreaCheckServlet, должна содержать:
\begin{enumerate}
    \item Таблицу, содержащую полученные параметры.
    \item Результат вычислений - факт попадания или непопадания точки в область.
    \item Ссылку на страницу с веб-формой для формирования нового запроса.
\end{enumerate}
Разработанное веб-приложение необходимо развернуть на сервере WildFly. Сервер должен быть запущен в standalone-конфигурации, порты должны быть настроены в соответствии с выданным portbase, доступ к http listener'у должен быть открыт для всех IP.
\section*{Итоговый результат}
Исходный код разработанной веб-страницы находится в \href{https://github.com/Armemius/ItmoStuff/tree/main/web/lab2}{данном репозитории} \\
\hfill\break
Ссылка на сайт на \href{https://se.ifmo.ru/~s368849/lab2/}{se.ifmo.ru}
\section*{Вывод}
Я разработал приложение, используя Java EE и научился работать с инструментарием, предоставляемым данной платформой
\end{document}