\documentclass[12pt,a4paper]{report}
\usepackage[T2A]{fontenc}
\usepackage[utf8]{inputenc}
\usepackage[russian]{babel}
\usepackage{graphicx, setspace, hyperref}

\usepackage[
top = 1.25cm, 
bottom = 2.0cm]{geometry}

\begin{document}
\begin{titlepage} 
	\centering
    % HEADER
	{
        \scshape
        Федеральное государственное автономное образовательное учреждение высшего образования
        \par
        \textbf{«Научно-образовательная корпорация ИТМО»}
        \par
        \vspace*{1cm}
        Факультет Программной Инженерии и Компьютерной Техники
        \par
    }
    % LOGO
    \vspace*{0.6cm}
    \includegraphics[width=\textwidth]{logo.png}
    % LAB INFO
    {
        \Large
        \textbf{Лабораторная работа по веб-программированию №3}
        \par
        \normalsize
        \vspace*{0.75cm}
        \textbf{Вариант 95762}
        \par
    }
    \vfill
    % СREDITS
    \hfill\begin{minipage}{\dimexpr\textwidth-7.8cm}
        \textbf{Выполнил:}\par
        Степанов Арсений Алексеевич\par
        \vspace*{0.15cm}
        \textbf{Группа:}\par
        P3209\par
        \vspace*{0.15cm}
        \textbf{Преподаватель:}\par
        Кулинич Ярослав Вадимович\par
    \end{minipage}
    \vfill
    Санкт-Петербург, \the\year{}г.
\end{titlepage}  
\section*{Цели}
Познакомиться с JavaServer Faces Framework, компонентами представленными библиотеками PrimeFaces и IceFaces, узнать что такое ORM и как с ними работать
\section*{Задание}
Разработать приложение на базе JavaServer Faces Framework, которое осуществляет проверку попадания точки в заданную область на координатной плоскости.\\
\hfill\break
Приложение должно включать в себя 2 facelets-шаблона - стартовую страницу и основную страницу приложения, а также набор управляемых бинов (managed beans), реализующих логику на стороне сервера.\\
\hfill\break
\textbf{Стартовая страница должна содержать следующие элементы:}
\begin{itemize}
    \item "Шапку", содержащую ФИО студента, номер группы и номер варианта.
    \item Интерактивные часы, показывающие текущие дату и время, обновляющиеся раз в 8 секунд.
    \item Ссылку, позволяющую перейти на основную страницу приложения.
\end{itemize}
\textbf{Основная страница должна содержать следующие элементы:}
\begin{itemize}
    \item Набор компонентов для задания координат точки и радиуса области в соответствии с вариантом задания. Может потребоваться использование дополнительных библиотек компонентов - ICEfaces (префикс "ace") и PrimeFaces (префикс "p"). Если компонент допускает ввод заведомо некорректных данных (таких, например, как буквы в координатах точки или отрицательный радиус), то приложение должно осуществлять их валидацию.
    \item Динамически обновляемую картинку, изображающую область на координатной плоскости в соответствии с номером варианта и точки, координаты которых были заданы пользователем. Клик по картинке должен инициировать сценарий, осуществляющий определение координат новой точки и отправку их на сервер для проверки её попадания в область. Цвет точек должен зависить от факта попадания / непопадания в область. Смена радиуса также должна инициировать перерисовку картинки.
    \item Таблицу со списком результатов предыдущих проверок.
    \item Ссылку, позволяющую вернуться на стартовую страницу.
\end{itemize}
\textbf{Дополнительные требования к приложению:}
\begin{itemize}
    \item Все результаты проверки должны сохраняться в базе данных под управлением СУБД PostgreSQL.
    \item Для доступа к БД необходимо использовать ORM Hibernate.
    \item Для управления списком результатов должен использоваться Application-scoped Managed Bean.
    \item Конфигурация управляемых бинов должна быть задана с помощью аннотаций.
    \item Правила навигации между страницами приложения должны быть заданы в отдельном конфигурационном файле.
\end{itemize}
\begin{center}
    \includegraphics*{graph.png}
\end{center}
\section*{Итоговый результат}
Исходный код разработанной веб-страницы находится в \href{https://github.com/Armemius/ItmoStuff/tree/main/web/lab3}{данном репозитории} \\
\section*{Вывод}
Я разработал приложение на Java EE с использованием JavaServer Faces Framework, набором компонентов библиотеки PrimeFaces, ORM Hibernate и научился работать с каждой из перечисленных технологий
\end{document}