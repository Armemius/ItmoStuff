\documentclass[12pt,a4paper]{report}
\usepackage[T2A]{fontenc}
\usepackage[utf8]{inputenc}
\usepackage[russian]{babel}
\usepackage{graphicx, setspace, amsmath}

\usepackage[
top = 1.25cm, 
bottom = 2.0cm]{geometry}

\begin{document}
\begin{titlepage} 
	\centering
    % HEADER
	{
        \scshape
        Федеральное государственное автономное образовательное учреждение высшего образования
        \par
        \textbf{«Научно-образовательная корпорация ИТМО»}
        \par
        \vspace*{1cm}
        Факультет Программной Инженерии и Компьютерной Техники
        \par
    }
    % LOGO
    \vspace*{0.6cm}
    \includegraphics[width=\textwidth]{logo.png}
    % LAB INFO
    {
        \Large
        \textbf{Практическая работа №6}
        \par
        \normalsize
        \vspace*{0.75cm}
        \textbf{Вариант 12}
        \par
    }
    \vfill
    % СREDITS
    \hfill\begin{minipage}{\dimexpr\textwidth-7.8cm}
        \textbf{Выполнил:}\par
        Степанов Арсений Алексеевич\par
        \vspace*{0.15cm}
        \textbf{Группа:}\par
        ТеорВер 2.4\par
        \vspace*{0.15cm}
        \textbf{Преподаватель:}\par
        Селина Елена Георгиевна\par
    \end{minipage}
    \vfill
    Санкт-Петербург, \the\year{}г.
\end{titlepage}  
\section*{Задание №1}
Дана выборка: $3.6\;3.9\;4.5\;3.8\;4.4\;4.9\;4.2\;3.8$ \\
\hfill\break
Построить доверительный интервал оценки генеральной средней при заданной доверительной вероятности $\gamma = 0.95$
\subsection*{Решение}
Размер выборки $n < 30$, поэтому воспользуемся следующей формулой:
$$\overline{X}-t_\frac{{\gamma + 1}}{2}(n-1)\frac{S}{\sqrt{n}}<m<\overline{X}+{t_\frac{\gamma + 1}{2}}(n-1)\frac{S}{\sqrt{n}}$$
Рассчитаем $\overline{X}$:
$$\overline{X}=\frac{1}{n}\sum_{i=1}^n x_i=\frac{31.1}{8}=4.1375$$
Рассчитаем $D(X)$:
$$D(X)=\frac{1}{n}\sum_{i=1}^n x_i^2-\overline{X}^2=\frac{138.31}{8}-17.1189=0.16985$$
Рассчитаем $S$:
$$S=\sqrt{\frac{n}{n-1}\cdot D(X)}=\sqrt{\frac{8}{7}\cdot 0.16985}=0.44058$$
Находим квантиль:
$$\gamma=0.99, n=8\qquad\frac{1 + 0.99}{2}=0.995\qquad t_{0.995}(7)=3.499$$
Найдём доверительный интервал:
$$4.1375-3.499\cdot\frac{0.44058}{2.64575} < m < 4.1375+3.499\cdot\frac{0.44058}{2.64575}$$
$$3.555 < m < 4.531$$
\section*{Задание №2}
$n=100$, $\overline{X}=748$, $\sigma=3.6$, $\gamma=0.95$\\
\hfill\break
Построить доверительный интервал оценки генеральной средней при заданной доверительной вероятности $\gamma = 0.95$
\subsection*{Решение}
Размер выборки $n > 30$, поэтому воспользуемся следующей формулой:
$$\overline{X}-\frac{t_\gamma\sigma}{\sqrt{n}} < m < \overline{X}+\frac{t_\gamma\sigma}{\sqrt{n}}$$
Найдём квантиль:
$$\gamma=0.95\qquad\frac{1+0.95}{2}=0.975\qquad t=1.96$$
Найдём доверительный интервал:
$$748-\frac{1.96\cdot 3.6}{10} < m < 748+\frac{1.96\cdot 3.6}{10}$$
$$747.2944 < m < 748.7056$$
\section*{Задание №3}
Распределение семян сорняков в выборках семян тимофеевки:\\
\hfill\break
\begin{tabular}{|c|c|c|c|c|c|c|c|c|c|c|}
    \hline
    Число семян сорняков & 0 & 1 & 2 & 3 & 4 & 5 & 6 & 7 & 8 & 9 \\
    \hline
    Число выборок & 3 & 17 & 26 & 16 & 18 & 9 & 3 & 5 & 0 & 1 \\
    \hline
\end{tabular}\\
\hfill\break
При заданном уровне значимости $\alpha=0.1$ проверить гипотезу о том, что выборка имеет распределение Пуассона
\subsection*{Решение}
Найдём выборочное среднее, чтобы оценить параметр $\lambda$:
$$\lambda=\frac{\sum_{i=0}^{n}(x_i\cdot i)}{\sum_{i=0}^{n}}=\frac{296}{98}=3.02$$
\hfill\break
\begin{center}
    \begin{tabular}{|c|c|c|c|}
        \hline
        $i$ & $n_i$ & $p_i$ & $n_i^*$ \\
        \hline
        0 & 3 & 0.04878 & 5 \\
        \hline
        1 & 17 & 0.14733 & 14 \\ % + 
        \hline
        2 & 26 & 0.22251 & 22 \\
        \hline
        3 & 16 & 0.22402 & 22 \\
        \hline
        4 & 18 & 0.16916 & 17 \\
        \hline
        5 & 9 & 0.10218 & 10 \\
        \hline
        6 & 3 & 0.05144 & 5 \\
        \hline
        7 & 5 & 0.02219 & 2 \\
        \hline
        8 & 0 & 0.00838 & 1 \\
        \hline
        9 & 1 & 0.00281 & 0 \\
        \hline
        $\sum$ & 98 & - & 98 \\
        \hline
    \end{tabular}
\end{center}
Объединим строки, которые не удовлетворяют условию $p_in\geq 9$, т.е. такие строки где $p_i < 0.092$
\begin{center}
    \begin{tabular}{|c|c|c|c|}
        \hline
        $i$ & $n_i$ & $n_i^*$ & $\frac{(n_i-n_i^*)^2}{n_i^*}$ \\
        \hline
        $\leq 1$ & 20 & 19 & 0.05263 \\
        \hline
        2 & 26 & 22 & 0.72727 \\
        \hline
        3 & 16 & 22 & 1.63636 \\
        \hline
        4 & 18 & 17 & 0.05882 \\
        \hline
        $\geq 5$ & 18 & 18 & 0.0 \\
        \hline
        $\sum$ & 98 & 98 & 2.47509 \\
        \hline
    \end{tabular}
\end{center}
Таким образом получаем, что $\chi^2_\text{набл}=2.47509$\\
\hfill\break
Найдём табличное значение $\chi^2_\text{крит}$:
$$\chi^2_\text{крит}=\chi^2_{1-\alpha}(k-l-1)=\chi^2_{0.99}(5-1-1)=\chi^2_{0.99}(3)=11.3$$

Получаем, что $\chi^2_\text{крит} > \chi^2_\text{набл}$
\end{document}