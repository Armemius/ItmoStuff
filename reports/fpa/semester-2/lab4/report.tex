\documentclass[12pt,a4paper]{report}
\usepackage[T2A]{fontenc}
\usepackage[utf8]{inputenc}
\usepackage[russian]{babel}
\usepackage{graphicx, setspace, longtable}

\usepackage[
top = 1.25cm, 
bottom = 2.0cm]{geometry}

\begin{document}
\begin{titlepage} 
	\centering
    % HEADER
	{
        \scshape
        Федеральное государственное автономное образовательное учреждение высшего образования
        \par
        \textbf{«Научно-образовательная корпорация ИТМО»}
        \par
        \vspace*{1cm}
        Факультет Программной Инженерии и Компьютерной Техники
        \par
    }
    % LOGO
    \vspace*{0.6cm}
    \includegraphics[width=\textwidth]{logo.png}
    % LAB INFO
    {
        \Large
        \textbf{Лабораторная работа по ОПД №4}
        \par
        \normalsize
        \vspace*{0.75cm}
        \textbf{Вариант 945}
        \par
    }
    \vfill
    % СREDITS
    \hfill\begin{minipage}{\dimexpr\textwidth-7.8cm}
        \textbf{Выполнил:}\par
        Степанов Арсений Алексеевич\par
        \vspace*{0.15cm}
        \textbf{Группа:}\par
        P3109\par
        \vspace*{0.15cm}
        \textbf{Преподаватель:}\par
        Ткешелашвили Нино Мерабиевна\par
    \end{minipage}
    \vfill
    Санкт-Петербург, \the\year{}г.
\end{titlepage}  
    \section*{Задание}
    По выданному преподавателем варианту восстановить текст заданного варианта программы и подпрограммы (программного комплекса), определить предназначение и составить его описание, определить область представления и область допустимых значений исходных данных и результата, выполнить трассировку программного комплекса.
    \section*{Программный комплекс}
    \begin{center}
        \begin{tabular}{|ccc|ccc||ccc|}
            \hline
            163: & + & 0200 & 171: & & 0740 & 6F0: & & AC01 \\
            \hline
            164: &   & EE19 & 172: & & 6E0B & 6F1: & & F203 \\
            \hline
            165: &   & AE15 & 173: & & EE0A & 6F2: & & 7E08 \\
            \hline
            166: &   & 0700 & 174: & & AE08 & 6F3: & & F004 \\
            \hline
            167: &   & 0C00 & 175: & & 0C00 & 6F4: & & F803 \\
            \hline
            168: &   & D6F0 & 176: & & D6F0 & 6F5: & & 4C01 \\
            \hline
            169: &   & 0800 & 177: & & 0800 & 6F6: & & 6E05 \\
            \hline
            16A: &   & 0740 & 178: & & 4E05 & 6F7: & & CE01 \\
            \hline
            16B: &   & 6E12 & 179: & & EE04 & 6F8: & & AE02 \\
            \hline
            16C: &   & EE11 & 17A: & & 0100 & 6F9: & & EC01 \\
            \hline
            16D: &   & AE0E & 17B: & & \{ZZZZ\} & 6FA: & & 0A00 \\
            \hline
            16E: &   & 0C00 & 17C: & & \{YYYY\} & 6FB: & & 0DF3 \\
            \hline
            16F: &   & D6F0 & 17D: & & \{XXXX\} & 6FC: & & 00AF \\
            \hline
            170: &   & 0800 & 17E: & & 0DF3 & & & \\
            \hline
        \end{tabular}
    \end{center}
    \section*{Анализ программы}
    \begin{longtable}{|c|c|c|c|}
        \hline
        Адр. & Код & Мнемоника & Комментарий \\
        \hline
        0x163 & 0200 & CLA & Очистка аккумулятора \\
        \hline
        0x164 & EE19 & ST R & AC $\rightarrow$ MEM(17E) \\
        \hline
        0x165 & AE15 & LD Z & MEM(17B) $\rightarrow$ AC \\
        \hline
        0x166 & 0700 & INC & Инкремент AC \\
        \hline
        0x167 & 0C00 & PUSH & AC на стек \\
        \hline
        0x168 & D6F0 & CALL 0x6F0 & Вызов подпрограммы \\
        \hline
        0x169 & 0800 & POP & Вершина стека в AC \\
        \hline
        0x16A & 0740 & DEC & Декремент AC \\
        \hline
        0x16B & 6E12 & SUB R & AC - MEM(17E) $\rightarrow$ AC \\
        \hline
        0x16C & EE11 & ST R & AC $\rightarrow$ MEM(17E) \\
        \hline
        0x16D & AE0E & LD Y & MEM(17C) $\rightarrow$ AC \\
        \hline
        0x16E & 0C00 & PUSH & AC на стек \\
        \hline
        0x16F & D6F0 & CALL 6F0 & Вызов подпрограммы \\
        \hline
        0x170 & 0800 & POP & Вершина стека в AC \\
        \hline
        0x171 & 0740 & DEC & Декремент AC \\
        \hline
        0x172 & 6E0B & SUB R & AC - MEM(17E) $\rightarrow$ AC \\
        \hline
        0x173 & EE0A & ST R & AC $\rightarrow$ MEM(17E) \\
        \hline
        0x174 & AE08 & LD X & MEM(17D) $\rightarrow$ AC \\
        \hline
        0x175 & 0C00 & PUSH & AC на стек \\
        \hline
        0x176 & D6F0 & CALL 6F0 & Вызов подпрограммы \\
        \hline
        0x177 & 0800 & POP & Вершина стека в AC \\
        \hline
        Адр. & Код & Мнемоника & Комментарий \\
        \hline
        0x178 & 4E05 & ADD R & AC + MEM(17E) $\rightarrow$ AC \\
        \hline
        0x179 & EE04 & ST R & AC $\rightarrow$ MEM(17E) \\
        \hline
        0x17A & 0100 & HLT & Остановка программы \\
        \hline
        0x17B & ZZZZ & - & Переменная Z \\
        \hline
        0x17C & YYYY & - & Переменная Y \\
        \hline
        0x17D & XXXX & - & Переменная X \\
        \hline
        0x17E & 0DF3 & - & Переменная R \\
        \hline
        \hline
        0x6F0 & AC01 & LD (SP+1) & MEM(SP+1) $\rightarrow$ AC \\
        \hline
        0x6F1 & F203 & BMI 3 & Переход на 6F5, если N=1 \\
        \hline
        0x6F2 & 7E08 & CMP A & Флаги по AC - MEM(6FB) \\
        \hline
        0x6F3 & F004 & BEQ 4 & Переход на 6F8, если Z=1 \\
        \hline
        0x6F4 & F803 & BLT 3 & Переход на 6F8, если AC < MEM(6FB) \\
        \hline
        0x6F5 & 4C01 & ADD (SP+1) & AC + MEM(SP+1) $\rightarrow$ AC \\
        \hline
        0x6F6 & 6E05 & SUB B & AC - MEM(6FC) $\rightarrow$ AC \\
        \hline
        0x6F7 & CE01 & JUMP 1 & IP + 2 $\rightarrow$ IP \\
        \hline
        0x6F8 & AE02 & LD A & MEM(6FB) $\rightarrow$ AC \\
        \hline
        0x6F9 & EC01 & ST (SP + 1) & AC $\rightarrow$ MEM(SP+1) \\
        \hline
        0x6FA & 0A00 & RET & Возврат из подпрограммы \\
        \hline
        0x6FB & 0DF3 & - & Переменная A \\
        \hline
        0x6FC & 00AF & - & Переменная B \\
        \hline
    \end{longtable}
    \section*{Анализ функции}
    Подпрограмма является функцией, которая принимает единственное значение и 
    считается следующим образом:\\
    \hfill\break
    $ 2x-B$, $x\in(-\infty,0)\cup(A, +\infty) $\\
    $ A $, $x\in [0, A]$\\
    \hfill\break
    Сама программа вычисляет следующее выражение:\\
    \hfill\break
    $f(X)+f(Y)-f(Z+1)$
    \section*{Вывод}
    Я научился работать и применять на практике стек и подпрограммы, что безусловно поможет мне в дальнейшем изучении принципов работы базовой ЭВМ и ассемблера
\end{document}